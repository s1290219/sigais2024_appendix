\section*{付録}
\paragraph{自動生成した標準語の文章:$D_1$}
昔々あるところに、おじいさんとおばあさんが住んでいました。ある日、おばあさんが川で洗濯をしていると、大きな桃が川から流れてきました。「なんて大きな桃だろう!」と驚いたおばあさんは、桃を家に持ち帰り、おじいさんと一緒に切ってみることにしました。 ところが、桃を切った途端、中から元気な男の赤ちゃんが現れました。おじいさんとおばあさんはとても驚きましたが、大喜びし、その子を「桃太郎」と名付けて育てることにしました。 桃太郎はすくすくと成長し、立派な若者になりました。そんなある日、桃太郎は鬼ヶ島に住む怖い鬼たちが村を荒らしていることを知り、鬼退治に行く決心をします。おじいさんとおばあさんは、旅立つ桃太郎のために、うまいきびだんごを作って持たせました。 桃太郎が旅をしていると、途中で犬、猿、キジに出会いました。桃太郎がきびだんごを分け与えると、みんな「鬼退治に協力するよ!」と言って、桃太郎の仲間になりました。 そして、ついに鬼ヶ島に到着した桃太郎と仲間たちは、力を合わせて鬼たちと戦い、見事に勝ちました。鬼たちは反省し、桃太郎に宝物を差し出しました。桃太郎は村へ戻り、その宝物でみんなを幸せにしました。 こうして、桃太郎と仲間たちは村で平和に暮らしましたとさ。めでたしめでたし。

\paragraph{文章$D_1$を方言話者が方言に翻訳した文章:$D_2$}
むがしむがしあるどごろに、じさまとばさまが住んでらっだず。ある日、ばさまが川で洗濯すさ行っだどぢ、おっちだ桃が川から流れてきたず。「なんたらおっちい桃だべ!」とたまげだばさまは、その桃を家さ持って帰って、じさまと一緒に切ってみることにしたず。そしたっきゃあ、桃を切ったっきゃあ、中がら元気だおどごわらすが出てきたず。じさまとばさまはほにたまげだども、うれしぐなって、そのわらしを「桃太郎」どなめゃあっこ付げで育てることにしたず。桃太郎はすくすくおがって、立派だ若者さなったず。そんなある日、桃太郎は鬼ヶ島さ住むこえゃあ鬼んどが村を荒らしてるごどをわがって、鬼退治さ行ぐ決心したず。じさまとばさまは、旅立つ桃太郎のために、うめゃあきびだんごをつぐっでもだせだず。桃太郎が旅しているどぢ、途中で犬と猿とキジに会ったず。桃太郎がきびだんごをわげでやっだら、みんな「鬼退治さ助けんべ!」とへって、桃太郎の仲間になったとさ。んで、とうとう鬼ヶ島さ着いだ桃太郎と仲間んどは、力合わせで鬼んどと戦って、見事に勝っだず。鬼んどは反省して、桃太郎さ宝物をわだしたず。みんなの期待に桃太郎は村さ戻って、その宝物でみんなを幸せにしたず。こうして、桃太郎と仲間んどは村で平和に暮らしたず。どっとはれ。

\paragraph{文章$D_2$をモデルが標準語に翻訳した文章:$D_3$}
むかしむかし、あるところに、おじいさんとおばあさんが住んでいました。ある日、おばあさんが川で洗濯をしに行ったとき、大きな桃が川から流れてきました。「なんて大きな桃なんだろう!」と驚いたおばあさんは、その桃を家に持って帰り、おじいさんと一緒に切ってみることにしました。すると、桃を切ったところ、中から元気な男の子が出てきました。おじいさんとおばあさんはとても驚きましたが、うれしくなって、その子を「桃太郎」と名付けて育てることにしました。桃太郎はすくすくと育ち、立派な若者になりました。そんなある日、桃太郎は鬼ヶ島に住む恐ろしい鬼たちが村を荒らしていることを知り、鬼退治に行く決心をしました。おじいさんとおばあさんは、旅立つ桃太郎のために、おいしいきびだんごを持たせてあげました。桃太郎が旅をしていると、途中で犬と猿とキジに会いました。桃太郎がきびだんごを分けてあげたところ、みんな「鬼退治を手伝おう!」と言って、桃太郎の仲間になりました。そして、ついに鬼ヶ島に着いた桃太郎と仲間たちは、力を合わせて鬼たちと戦い、見事に勝利しました。鬼たちは反省して、桃太郎に宝物を差し出しました。みんなの期待に応えて、桃太郎は村に戻り、その宝物で村の人々を幸せにしました。こうして、桃太郎と仲間たちは村で平和に暮らしました。めでたしめでたし。
